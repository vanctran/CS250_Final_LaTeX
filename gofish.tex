\documentclass{article}
\usepackage{bigpage}
\usepackage{blockpar}
\usepackage{blocktitle}
\usepackage{czt}

\title{CS 250 Final Project - GoFish Specification}
\author{Van Tran / vct}
\date{2016-3-7}

\begin{document}
\maketitle

\section*{Introduction}
This is the Z specification for the card game, GoFish.
However, be aware that this is not the conventional version
of the game. There will be minor modifications to the game
that will be mentioned throughout this specification.

\section*{Rules}
\begin{figure}[h]
\begin{quotation}
1. Game has two players \\
2. General deck of 52 cards \\
3. Game starts with each player dealt 7 cards from the
shuffled deck. The rest are stockpiled for drawing. \\
4. The person who didn't deal the cards gets the first turn. \\
5. During a player's turn (also known as fishing). The player
asks the opposing player if they have a certain card (suit doesn't matter).
The player fishing must have at least one card in the rank of the card request. \\
6. The opposing player will then hand over the card that matches
the request. However, if the defending player does not have the card,
they will say 'GoFish', and the fishing player will draw a card
from the stockpile deck. If the deck is empty, there is no drawing phase. \\
7. At any time that any of the players obtain two of the same cards,
either during the drawing (go fishing), dealing, or successful request,
the pair of cards can be removed from the hand. \\
8. Once either of the scenarios has passed, the roles will be swapped,
and the defender will become the fisher and the fisher is now the defender. \\
9. At any moment in time, if a player's hand is empty, and there are still
cards in the deck, then they are rewarded with 5 more cards drawn from the deck. \\
10. The turns will continue until there are no more cards in play, in both
the deck and both players hands. \\
11. The winner will be the player with more matched pairs. \\
\end{quotation}
\end{figure}

There are two players in this game. Player 1 will always start the first turn.
\begin{zed} 
    PLAYERS ::= player1 | player2
\end{zed}

Of course, we must have cards to play the game.
Numbers 1, 11, 12, and 13 represents, Ace, Jack, Queen, and King, respectively.
There are four suits of each card
\begin{zed}
    SUIT ::= spade | club | diamond | heart \\
    RANK == 1\upto13 \\
    CARD == RANK \cross SUIT \\
    HAND == \power CARD \\
    DECK == \seq CARD \\
\end{zed}

\clearpage

The deck is a stack of 52 cards. 
\begin{schema}{Deck}
    deck : DECK
    \where
    \#deck \leq 52
\end{schema}

Deck is shuffled before the game begins. A deck has 52 cards in it at the
start of the game. The shuffling process is not specified in Z notation
to help simplify the specification.
\begin{schema}{InitDeck}
    Deck
    \where
    \#deck = 52
\end{schema}

Each player has an identity (player 1 or player 2) to
differentiate between. Each player has a hand of cards, with
no more than 13 cards because 14 would mean you have a match.
A player will have a "score" pile of cards as well to hold
matching pairs of cards that count as 1 point per pair.

\begin{schema}{Player}
    identity : PLAYERS \\
    hand : HAND \\
    score : \nat
    \where
    score \leq 26 \\
    \#hand \leq 13
\end{schema}

If a player is allowed to draw a card from the deck, it must be drawn from the top.
\begin{axdef}
    draw : DECK \fun \power CARD
    \where
    \forall j : DECK @ draw(j) = (\{head~j\})
\end{axdef}

The card is removed from the deck once it is drawn. Then is added to the player's hand.
\begin{schema}{Draw\_Card}
    \Delta Player \\
    \Delta Deck
    \where
    \#deck \neq 0 \\
    \forall i : \power CARD | \\
    \t1 i = draw(deck) @ \\
    \t2 hand' = hand \cup i \land deck' = tail~deck
\end{schema}

The player will begin with a score of 0, an empty hand
and is identified as player1 or player2
\begin{schema}{Identify}
    Player
    \where
    hand = \emptyset \\
    score = 0
\end{schema}

Each player will be dealt 5 cards at the beginning of the game.
\begin{schema}{Deal}
    \Xi Player
    \where
    \#hand < 5
\end{schema}

\begin{zed}
    Deal\_Cards == Deal \land Draw\_Card\\
    InitPlayer == Identify \semi Deal\_Cards
\end{zed}

The player turn in which a player is either a fisher
or a defender. The fisher will request a card from the defender,
who will then search their hand and give up their card if it exists.
\begin{schema}{Player\_Turn}
    fisher, defender : Player
    \where
    fisher.identity \neq defender.identity\\
    \#fisher.hand \neq 0 \land \#defender.hand \neq 0
\end{schema}

A response from the defender when the fisher guesses
\begin{zed}
RESPONSE ::= go\_fish | correct
\end{zed}

\begin{axdef}
    guessing : CARD \cross HAND \fun RESPONSE
    \where
    \forall i : CARD; j : HAND @ \\
    \t1 (i \in j \iff guessing(i, j) = correct) \lor \\
    \t1 (i \notin j \iff guessing(i, j) = go\_fish)
\end{axdef}

When the user makes an incorrect guess, they must draw from
the deck. Nothing happens to the defender in this situation.
\begin{schema}{Guess\_Wrong}
    \Xi Player\_Turn \\
    guess? : CARD 
    \where
    guessing(guess?, defender.hand) = go\_fish
\end{schema}
 
However, if the deck were to be empty, a the drawing phase is skipped.
During this state, the deck cannot be drawn from.
\begin{schema}{Empty\_Deck}
    \Xi Deck \\
    \Xi Player
    \where
    deck = \emptyset \iff \#deck = 0
\end{schema}

The fisher gets the correct guess then the defender must
give up the card that was found
\begin{schema}{Guess\_Correct}
    \Xi Player\_Turn \\
    guess? : CARD \\
    retrieve! : CARD
    \where
    guessing(guess?, defender.hand) = correct\\
    retrieve! = guess?
\end{schema}

The defender gives up the card that the fisher correctly guessed
\begin{schema}{Lose\_Card}
    \Delta Player \\
    guess? : CARD \\
    lost\_card! : CARD
    \where
    hand' = hand \setminus \{guess?\} \\
    lost\_card! = guess?
\end{schema}

The fisher now gets that card from the defender
\begin{schema}{Get\_Card}
    \Delta Player \\
    gained\_card? : CARD
    \where
    hand' = hand \cup \{gained\_card?\}
\end{schema}

The player has found a matching pair of cards and then proceeds
to remove those cards from their hand.
\begin{schema}{Pair}
    \Delta Player \\
    card1?,card2? : CARD
    \where
    first(card1?) = first(card2?) \\
    hand' = hand \setminus \{card1?\} \cup \{card2?\} \\ 
    score' = score + 1
\end{schema}

If the player's hand is empty while the game is still going...
\begin{schema}{Empty\_Hand}
    \Xi Player
    \where
    hand = \emptyset
\end{schema}

The player is allowed to draw one card from the deck. The game should already
be over if the deck is empty at this point.
\begin{zed}
    Empty\_Draw == Empty\_Hand \land Draw\_Card
\end{zed}


At the end of a turn, the users would switch places
\begin{schema}{End\_Turn}
    \Delta Player\_Turn
    \where
    fisher' = defender \\
    defender' = fisher
\end{schema}


These are the possible outcomes of the game. A player can
either win, lose, or draw. If a player has more points than their
opponent, it is considered a victory. If a player has less points
than their opponent, it is considered a loss. If both players have
an equal number of points, the result is a tie.
\begin{zed}
    RESULT ::= win | lose | tie
\end{zed}

\begin{axdef}
    outcome : Player \cross Player \fun RESULT
    \where
    \forall p1, p2 : Player @\\
    \t1 (p1.score > p2.score \land outcome(p1,p2) = win) \lor \\
    \t1 (p1.score = p2.score \land outcome(p1,p2) = tie) \lor \\
    \t1 (p1.score < p2.score \land outcome(p1,p2) = lose)
\end{axdef}

To win, there must be no more cards left in the deck as well as
in both players hands. The winning player should always have more
points than the losing player.
\begin{schema}{Outcome}
    Deck \\
    Player\_Turn\\
    winner : \power Player
    \where
    \#deck + \#fisher.hand + \#defender.hand = 0 \\
    \forall i : RESULT | i = outcome(fisher, defender) @\\
    (i = win \land winner = \{fisher\}) \lor \\
    (i = lose \land winner = \{defender\}) \lor \\
    (i = tie \land winner = \{fisher\} \cup \{defender\})
\end{schema}

During the guessing phase, the fisher can either guess correct or wrong
\begin{zed}
    Incorrect == Guess\_Wrong \land (Draw\_Card \lor Empty\_Deck)\\
    Correct == (Guess\_Correct \pipe Lose\_Card) \pipe Get\_Card \\
    Guess == Incorrect \lor Correct
\end{zed}

The game of GoFish! My composition symbol doesn't seem to be working
\begin{zed}
    GoFish == InitPlayer~' \semi Player\_Turn \semi Guess \lor Outcome\\
\end{zed}
    
\end{document}
