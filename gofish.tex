\documentclass{article}
\usepackage{bigpage}
\usepackage{blockpar}
\usepackage{blocktitle}
\usepackage{czt}

\title{CS 250 Final Project - GoFish Specification}
\author{Van Tran / vct}
\date{2016-3-7}

\begin{document}
\maketitle

\section*{Introduction}
This is the Z specification for the card game, GoFish.
However, be aware that this is not the conventional version
of the game. There will be minor modifications to the game
that will be mentioned throughout this specification.

\section*{Rules}
\begin{figure}[h]
\begin{quotation}
1. Game has two players \\
2. General deck of 52 cards\\
3. Game starts with each player dealt 7 cards from the
shuffled deck. The rest are stockpiled for drawing. \\
4. The person who didn't deal the cards gets the first turn. \\
5. During a player's turn (also known as fishing). The player
asks the opposing player if they have a certain card (suit doesn't matter).
The player fishing must have at least one card in the rank of the card request.\\
6. The opposing player will then hand over the card that matches
the request. However, if the defending player does not have the card,
they will say 'GoFish', and the fishing player will draw a card
from the stockpile deck. If the deck is empty, there is no drawing phase.\\
7. At any time that any of the players obtain two of the same cards,
either during the drawing (go fishing), dealing, or successful request,
the pair of cards can be removed from the hand.
8. Once either of the scenarios has passed, the roles will be swapped,
and the defender will become the fisher and the fisher is now the defender.\\
9. The turns will continue until there are no more cards in play, in both
the deck and both players hands.\\
10. The winner will be the player with more matched pairs. \\
\end{quotation}
\end{figure}

There are two players in this game. Player 1 will always start the first turn.
\begin{zed} 
    PLAYERS ::= player1 | player2\\
\end{zed}

Of course, we must have cards to play the game.
Numbers 1, 11, 12, and 13 represents, Ace, Jack, Queen, and King respectively.

There are four suits of each card
\begin{zed}
    SUIT ::= spade | club | diamond | heart \\
    RANK == 1\upto13 \\
    CARD == RANK \cross SUIT \\
    DECK == \power CARD
\end{zed}

\begin{schema}{Deck}
    deck : DECK \\
    deck\_count : \nat
    \where
    \#(deck) \leq 52 \\  
    deck\_count \leq 52
\end{schema}

Deck is shuffled before the game begins.
\begin{schema}{Init\_Deck}
    \Delta Deck
    \where
    \#(deck) = deck\_count' = 52 \\
\end{schema}

Each player has an identity (player 1 or player 2) to
differentiate between. Each player has a hand of cards, with
no more than 13 cards because 14 would mean you have a match.
A player will have a "score" pile of cards as well to hold
matching pairs of cards that count as 1 point per pair.

\begin{schema}{Player}
    identity : PLAYERS \\
    hand : \power CARD \\
    hand\_count : \nat \\
    score : \nat \\
    \where
    score \leq 26 \\
    \#(hand) \leq 13 \land hand\_count \leq 13 \\
\end{schema}

The player will begin with a score of 0, an empty hand
and is identified as player1 or player2
\begin{schema}{Init\_Player}
    \Delta Player \\
    player\_number? : PLAYERS
    \where
    identity' = player\_number?\\
    hand' = \emptyset \\
    score' = 0 \\
\end{schema}

Draw function will return a card
\begin{axdef}
    draw : DECK \fun \power CARD\\
    \where
    \forall j : DECK @ draw(j) = (front~j)\\
\end{axdef}

The game will begin with each player being dealt 5 cards each
\begin{schema}{Dealer}
    \Delta Player \\
    \Delta Deck \\
    \where
    deck\_count' = deck\_count - 1\\  
    hand\_count' = hand\_count + 1\\
    \forall i : \power CARD | i = draw(deck) @
    \t1 hand' = hand \cup i \land
    \t1 deck' = deck \setminus i\\ 
\end{schema}

Cards will have to be drawn from the deck throughout the game.
Cannot draw anymore cards when the deck no longer has cards.
\begin{schema}{Draw\_Card}
    \Delta Player\\
    \Delta Deck\\
\end{schema}

You cannot draw from an empty deck
\begin{schema}{Empty\_Deck}
    \Xi Deck
    \where
    deck\_count = 0 \\
    deck = \emptyset
\end{schema}

The player turn is a state in which a player is either a fisher
or a defender. The fisher will request a card from the defender,
who will then search their hand and give up their card if it exists.
\begin{schema}{Player\_Turn}
    \Xi Deck \\
    fisher, defender : Player
    \where
    fisher.identity \neq defender.identity\\
    fisher.hand\_count \neq 0\\
    defender.hand\_count \neq 0\\ 
\end{schema}

The state at which the fisher gets the correct guess
\begin{schema}{Guess\_Correct}
    \Delta Player\_Turn \\
    guess? : CARD \\
    \where
    guess? \in defender.hand \\
\end{schema}

The player has found a matching pair of cards and then proceeds
to remove those cards from their hand.
\begin{schema}{Pair}
    \Delta Player \\
    card1?,card2? : CARD
    \where
    card1? = card2? \\
    hand' = hand  \\ 
    hand\_count' = hand\_count - 2 \\
    score' = score + 1
\end{schema}

\begin{schema}{End\_Turn}
    \Delta Player\_Turn
    \where
    fisher' = defender \\
    defender' = fisher
\end{schema}

These are the possible outcomes of the game. A player can
either win, lose, or draw. If a player has more points than their
opponent, it is considered a victory. If a player has less points
than their opponent, it is considered a loss. If both players have
an equal number of points, the result is a tie.

\begin{zed}
    RESULT ::= win | lose | tie
\end{zed}

\begin{axdef}
    outcome : Player \cross Player \fun RESULT
    \where
    \forall p1, p2 : Player @\\
    \t1 (p1.score > p2.score \land outcome(p1,p2) = win) \lor \\
    \t1 (p1.score = p2.score \land outcome(p1,p2) = tie) \lor \\
    \t1 (p1.score < p2.score \land outcome(p1,p2) = lose)
\end{axdef}

To win, there must be no more cards left in the deck as well as
in both players hands. The winning player should always have more
points than the losing player.
\begin{schema}{Victory}
    \Xi Deck \\
    winner, loser : Player \\
    \where
    (deck\_count + winner.hand\_count + loser.hand\_count) = 0\\
    outcome(winner, loser) = win\\
\end{schema}

\end{document}
